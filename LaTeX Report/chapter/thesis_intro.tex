\chapter{Introduction}
	\label{chap:intro}
	
	\textbf{As part of our Masters of Science accreditation, we must complete a research thesis. In has been decided upon to create an educational game centred around Machine Learning (ML), due to the authors desire to gain a deeper understanding of ML and their previous experiences as being a secondary school teacher.} %% Taken from Spec
	
	\section{Overview of the Problem}
		\label{sec:overview_of_problem}
		As every day goes by, machine learning is becoming more and more a part of our everyday lives. From voice assistance, predictive text suggestions, language translations, and even suggesting times to leave your home to get to your destination on time. Machine learning is very much ingrained in the tech that we use. However, machine learning gets perceived as a black box, a form of computer magic, where these unique algorithms, created by super-intelligent people, do some voodoo magical unknown thing. These naive views have led to a lot of misconceptions about the topic. Some misconceptions people have about Artificial Intelligence (AI) and ML is that 'AI does not need humans' and that 'AI is dangerous' \cite{quora5misconcepts}. Other misconceptions about AI and ML is that both concepts are very new, and the designed has got based on a human's brain. However, AI and ML is a technique that has been around for a long time. It is nowhere near the same as a human brain, even at the fundamental level. Another big misconception is that AI is smarter than humans and that the ML robots will come and destroy the humans. In result, leading to wiping out humanity and all life forms. However, while AI can be better at performing specific tasks, they are not genetically more intelligent than humans. AI will only do what it gets told to do, nothing more \cite{quora5misconcepts}.
		
		On the other hand, instead of fearing AI and ML, there is a lot of things that it is currently doing to help humankind and make things safer. For example, the RAC, one of the UK's largest motoring organisations, aims to try and detect low-speed car crashes. They do this by developing an onboard crash sensing system that uses advanced machine learning algorithms to detect low-speed collisions and distinguish these events from more common driving events, such as driving over speed bumps or potholes. Independent tests showed the RAC system to be 92\% accurate in detecting test crashes, allowing them to be able to enable rapid response to roadside incidents \cite{matlanintrotoml}.
	
	\section{Overview of the Solution}
		\label{sec:overview_of_solution}
		Our overall aim for the project is to research current techniques currently discussed within academia and methods that are used by developers and creators within the public domain. This research intention is to help us develop and create a fun and educational game about ML. The players will be, at the core of the solution, playing a game that interacts with different ML models. The player(s) will be manipulating the game board and data points to affect the decision boundary, or to figure out where the decision boundary or centre of the cluster is. The solution will get created by using  GUI libraries and will have many different algorithms in the background, contributing to the main game mechanics. The aim is to achieve this by using libraries like SKLearn \cite{sklearn_api} and Tensorflow \cite{tensorflow2015-whitepaper}, to name a few.
	
		
	
	%\section{Motivations}
	%\label{sec:intro_motivation} 
	
	
		
	%\section{Aims}
	%	\label{sec:aims}
		
	
	\subsection{Aims \& Objectives}
		\label{sec:intro_objective} 
		The proposed solution aims to create a fully interactive game that users will find fun and engaging, with elements of a scientific series game in certain areas within it. While, at the core of it, still providing a level of education to teach the players what the different machine learning algorithms are and how they fundamentally work. From our experience of being a teacher, learning has the most impact when the learner gets the chance to be able to fully interact with the learning subject content and see how it work first hand, rather than just being told about it and how it works. 
		
		We aim to create an educational game, that will help inform users what ML is and what it does. Which will aspire to, as a result, demystify the myths and misconceptions people have about ML and AI.
	
	
	\section{Contributions} 
		\label{sec:intro_contribs} 
		
		The main contributions of this work can be seen as follows:
		
		\begin{description}	
		
			\item[$\bullet$ A written thesis explaining the steps and stages of the development of the project]\hfill
			
			A document that's intention is to partner and explain aspects of the final application, explain decisions made and explain the research discovered to influence decisions.  
			
			\item[$\bullet$ An education game application about machine learning]\hfill
			
			An application created that allows the player or user to interact with, and manipulate, different machine learning models. Additional content, in the form of a website, has additionally been provided and hosted online. This supplementary content is to help with the teaching and learning of the main ML concepts used within the application.
			
			
		\end{description}
	
	\section{Thesis Overview}  
		\label{sec:intro_thesis_overview} 
		We will first look into the background literature related to this project. Looking into educational games, with relations to gamification generally and within education, while also looking at example applications already presented. We will also be looking into machine learning within this section and looking into its fundamentals. The fundamentals will involve looking into the required data, functions and dimensionality. While also looking into the different learning types of supervised and unsupervised learning.
		
		Additionally, we will look at all the algorithms and models that intend to get implemented within the application. These include linear regression, logistic regression, k nearest neighbour, SVM, PCA, LDA, GMM, k-means, neural networks, [add the rest]. We will also be looking at how machine learning gets currently presented within education and how the concepts are presently getting taught and any educational games related to machine learning.
		
		We will then go into the methodology of the project. This section will be explaining the overview of the application and its design and giving an overview of specific game components. While also providing the intended method for evaluating the application.
		
		Next, we will be looking at how the application got implemented, explaining the languages and frameworks used. As well as the intricacies required to get each section of the application working as intended. 
		
		The final stages will be evaluating the results of the user study and the progress of the project overall. Additionally, a conclusion and a discussion reflecting on how the project went overall will get presented. At the same time, we are presenting any possible future work that could get done with the project.
	
	