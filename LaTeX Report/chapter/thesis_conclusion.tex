\chapter{Evaluation}
	Within this section, we will evaluate the results from the user study. We will also assess the outcome of the app based on our initial intentions and the original plan created within our specification report for our intended approach to the project. We will also discuss and reflect on the application itself, exploring functionality that could not get implemented and areas that could get expanded. Additionally, closing this section with a general reflection of the project. This reflection involves a summary of how the application turned out by the end of the allocated time frame. 

	\section{Evaluation of User Study and Approach}
	[Need results]	


\chapter{Conclusion \& Discussion}
	\section{Reflection on the application}
		Although a lot of features got implemented within the application, there are a few features and little functionality that was, unfortunately, implemented within the application. One of the main pieces of functionality that are missing from the application is the 'Award Zone' and the overarching gamification unlockables and features. Although the application has some gamification techniques. For example, user competition and instant feedback. However, our intention was always for the application to have many more gamification features. These features would have been within every stage of the application with the LZ being the main hub displaying the users progress. 

		We intended for the use of progress bars, unlockable badges and unlockable features get used within the application. Some of the unlockable features intended were to, when the user completed certain milestones, to unlock example code of the different ML models within the learning zone and links to future reading and academic papers for the ML model. Another feature was that some models in the Free Play Zone and the Game area would be unlocked once the player had accomplished specific tasks.

		Another feature that did not get implemented was to present a pregame screen to the users. This screen was going to allow the players to be able to set up the game mechanics that were selected by the player. These game choices included: choosing a gameplay playlist, for example, territory or pin the data point;  how many rounds within the game the players wanted; setting the difficulty setting to the game; selecting the number of players. Another feature intended for the game area was to have more ML models implemented, especially all of the models available in the learning zone. However, one initial model that got implemented in the learning zone but did not get implemented in the game zone was the neural network model for a territory style gameplay. This style of game intention was to add extra variation and variety to the player. A missing feature that there was an intention for the game to have is a single-player mode. Within the free play area, the design was to have additional models available for the players to interact with, and these models being kNN and logistic regression, to match up with the content in the learning zone. A feature that did get implemented, but to the desired consistency, was the level of options available to the user when interacting with the model's parameters and data options. While specific models had this implemented, a consistent level did not get established between all of the models resulting in varying amounts of interactivity. 

		An area where the application can get expanded is to add the feature of extra players. Although the application currently supports two players, the ability to add three or four players would add additional functionality and competition within the game area. An area that could get expanded is to have a settings option. To allow players with colour blindness change the colour scheme, as this will make sure that they can access as much of the application as possible. An additional area that could get expanded ist the ability for the quiz questions to appear random and the answer get displayed in random order. Therefore, making sure the players not just to learn the locations of the answers but force them to read the options and think about the answers.

		[Any other suggestions?]

	\section{Reflection on the evaluation}
		What did users find good/bad etc.
		[Need results]


	\section{Reflection of the Project Development}
		Overall we found that the project did follow the planned schedule pretty well. However, there were several bumps along the way. The first bump in the road happened due to an initial incorrect choice of GUI framework early on. We initially intended to use Pygame as the GUI framework, as the application's main intention was to create a game. After committing time, as planned in the project Gantt chart, to learning the intended GUI library better, after we had no previous experience of Pygame before. It got realised that, although the library would be good at creating game screens and menus, it was unable to develop or support some essential libraries or features that are required to make all the aspects of the application. It then got decided to use the GUI library that got used on the implementation of the application. However, this led to less time getting allocated to learning the library. Therefore it led to a lot of aspects of the library getting learnt along the way. Which, we believe, is shown within some elements with the application. Resulting in the application not being as smooth with its layout and presentation as intended. For example, the screen layout has not got optimised for the changing screen sizes, even though resizing gets supported within the code. Just not as elegantly as we would have hoped.

		Although prototypes would get created effectively using either Jupyter Notebooks, for ML models and Matplotlib visualisations, to help with figuring out the initial logic required to make the models. However, when trying to convert these prototypes, at first into the GUI using Matplotlib backend handlers to help implement the visualisation graphs in the GUI, a lot of issues first arose. These issues were down to the GUI, even though it could support Matplotlib, it could not support the library in the same way that we have prior experience of using it. Therefore, it forced us into learning the required backend library and rewriting how some aspects of the code worked, for the graphs to get displayed to the user. However, once this got first implemented, the other models were more comfortable to develop afterwards. 

		Another decision that changed from what got initially planned to what happened when the application was getting implemented. This change was it was intended for the application to have a model implemented within the Freeplay area, and then straight away into the game area. However, it then got decided to implement as many models within the Freeplay area first, when after several models had got created, then add them to the game area. This decision got made because we thought it would be a better decision to have a complete section of the application, like the learning zone if we had run out of time due to unforeseen circumstance, then to have two areas be half done. Additionally, it got also decided because a lot of the functionality required for the free play area would become the foundation of the main game section.

		Although some initial intentions and ideas were intended only to get implemented within the application if there was enough time available, one key feature that was hoped for but ultimately did not get implemented was the full-on gamification techniques. These would have been within the Awards Zone. This feature always intended to get added after all the other critical aspects of the application. However, due to developing of the GUI taking longer to do than initially planned, it had a knock-on effect and didn't leave much time for the feature to get added. Although aspects of gamification did get implemented within other areas of the application, having a real awards area we believe would have added even more motivation for players to return. 

		A huge unforeseen risk that occurred was the pandemic we are finding ourself in today, Covid-19. Not only did this impact on being able to conduct a user study in the intended way planned, but it also has had massive unplanned knock-on effects. For example, with lockdowns being enforced by the government and then moving restricts being put in place. These have resulted in everyday activities, not being the same, which has also had an immense impact on energy levels and focus, due to lack of active activities happening and social activities getting cancelled.
		
	\section{Future Work} % (maybe even pie-in-the-sky stuff) (Min: 500)
		Future work for this application is that online functionality could get included and implemented. Allowing players to be able to face off against each other online, allowing the game element to be accessible to all, without having to have someone with you.
		
		Another area where the application could get adapted in the future is to have the quiz questions get stored in a cloud database, to allow quiz questions to be remotely updated. Therefore every time the player takes a quiz, they will be updated continuously without any updates or downloads required. Not only would this allow for fresh new content to keep the players interested, but it will also allow for when the content of the learning zone's webpages gets an update, or new models added, for the questions within the quiz to get updated as well.
		
		As most people use tablets and mobile devices, future work that could get done to the application is to create a web GUI to the game and host the entire application online. Through making the application web-based, the application could also then be made natively on mobile devices like iOS and Android and published through their respective app stores. Though, this will require the players always to have a constant internet connection. It will, however, allow for the players to be able to access the application, from anywhere at any time, in a manner more suited to them. Which, ultimately is the most important thing, the players are interacting with the application learning about the different ML models.
		
		Potential future work to the application is to allow players to upload their own generated datasets to the application within the free play area. This additional feature would be expanding off the serious game gamification notion of genuinely letting the users be able to interact with the data and get a vibe of the data, all within the sandbox of the application. 
%\chapter{Conclusions and Future Work}
%\label{chap:conclusion}

%In this document we have demonstrated the use of a LaTeX thesis template which can produce a professional looking academic document. 

%\section{Contributions} 
%\label{sec:conclusion_contributions}

%The main contributions of this work can be summarized as follows:
%\begin{description}	

%	\item[$\bullet$ A LaTeX thesis template]\hfill
	
%	Modify this document by adding additional top level content chapters. These descriptions should take a more retrospective tone as you include summary of performance or viability. 
	
	
%	\item[$\bullet$ A typesetting guide of useful primitive elements]\hfill
	
%	Use the building blocks within this template to typeset each part of your document. Aim to use simple and reusable elements to keep your document neat and consistently styled throughout.
	
%	\item[$\bullet$ A review of how to find and cite external resources]\hfill
		
%	We review techniques and resources for finding and properly citing resources from the prior academic literature and from online resources. 
	
%\end{description}

%\section{Future Work}
%\label{sec:conclusion_future_work}

%Future editions of this template may include additional references to Futurama.

	\section{Summary and closing comments}