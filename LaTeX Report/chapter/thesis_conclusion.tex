\chapter{Evaluation and Conclusion}
Within this section, we will evaluate the results from the user study. We will also assess the outcome of the app based on our initial intentions and the original plan created within our specification report for our intended approach to the project. We will also discuss and reflect on the application itself, exploring functionality that could not get implemented and areas that could get expanded. Additionally, closing this section with a general reflection of the project. This reflection involves a summary of how the application turned out by the end of the allocated time frame. 

\section{Evaluation of User Study and Approach}
[Need results]	

\section{Conclusion \& Discussion}
\subsection{Reflection on the application}
Although a lot of features got implemented within the application, there are a few features and little functionality that was, unfortunately, implemented within the application. One of the main pieces of functionality that are missing from the application is the 'Award Zone' and the overarching gamification unlockables and features. Although the application has some gamification techniques. For example, user competition and instant feedback. However, our intention was always for the application to have many more gamification features. These features would have been within every stage of the application with the LZ being the main hub displaying the users progress. 

We intended for the use of progress bars, unlockable badges and unlockable features get used within the application. Some of the unlockable features intended were to, when the user completed certain milestones, to unlock example code of the different ML models within the learning zone and links to future reading and academic papers for the ML model. Another feature was that some models in the Free Play Zone and the Game area would be unlocked once the player had accomplished specific tasks.

Another feature that did not get implemented was to present a pregame screen to the users. This screen was going to allow the players to be able to set up the game mechanics that were selected by the player. These game choices included: choosing a gameplay playlist, for example, territory or pin the data point;  how many rounds within the game the players wanted; setting the difficulty setting to the game; selecting the number of players. Another feature intended for the game area was to have more ML models implemented, especially all of the models available in the learning zone. However, one initial model that got implemented in the learning zone but did not get implemented in the game zone was the neural network model for a territory style gameplay. This style of game intention was to add extra variation and variety to the player. A missing feature that there was an intention for the game to have is a single-player mode. Within the free play area, the design was to have additional models available for the players to interact with, and these models being kNN and logistic regression, to match up with the content in the learning zone. A feature that did get implemented, but to the desired consistency, was the level of options available to the user when interacting with the model's parameters and data options. While specific models had this implemented, a consistent level did not get established between all of the models resulting in varying amounts of interactivity. 
%\chapter{Conclusions and Future Work}
%\label{chap:conclusion}

%In this document we have demonstrated the use of a LaTeX thesis template which can produce a professional looking academic document. 

%\section{Contributions} 
%\label{sec:conclusion_contributions}

%The main contributions of this work can be summarized as follows:
%\begin{description}	

%	\item[$\bullet$ A LaTeX thesis template]\hfill
	
%	Modify this document by adding additional top level content chapters. These descriptions should take a more retrospective tone as you include summary of performance or viability. 
	
	
%	\item[$\bullet$ A typesetting guide of useful primitive elements]\hfill
	
%	Use the building blocks within this template to typeset each part of your document. Aim to use simple and reusable elements to keep your document neat and consistently styled throughout.
	
%	\item[$\bullet$ A review of how to find and cite external resources]\hfill
		
%	We review techniques and resources for finding and properly citing resources from the prior academic literature and from online resources. 
	
%\end{description}

%\section{Future Work}
%\label{sec:conclusion_future_work}

%Future editions of this template may include additional references to Futurama.

\chapter{Summary and Final Comments}